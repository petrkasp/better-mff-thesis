\chapter{Parallel corpus format}
\label{parallel_corpus_format}

The pipeline uses a custom XML format for parallel corpora with SynSemClass annotations. See \cref{lst:generated_corpus} and \cref{lst:annotated_corpus} for an example.

The top tag is \texttt{sentences} and it is filled with individual \texttt{sentence} tags. The \texttt{sentence} tags have an \texttt{id} attribute that denotes the position of the sentence in the original corpus. Each \texttt{sentence} tag has three child tags: \texttt{text}, \texttt{source} and \texttt{verbs}.

The \texttt{text} tag contains the text of the sentence in the target language and the \texttt{source} tag contains the text of the sentence in the source language. They are followed by a \texttt{verbs} tag that contains instances of verbs in the target language in the form of \texttt{verb} tags.

Each \texttt{verb} tag contains a \texttt{class} attribute for the SynSemClass class that is not yet assigned by the pipeline and should be decided upon by the annotator. The SynSemClass class names are in the form of \texttt{vec} followed by 5 digits, e.g., \texttt{vec00017}. As we manually annotated the verbs into a constantly developing and as such, yet unfinished, class set of the SynSemClass ontology, we inevitably came across verbs without an assigned class in the ontology. In such cases, when we believed the verb sense is not yet annotated in the ontology, we inserted \texttt{TBD} instead of the class name. The \texttt{verb} tag also contains a \texttt{lemma} attribute generated by the pipeline.

Inside each \texttt{verb} tag, a \texttt{mark} tag has the sentence in the target language marked with the hat symbol (\textasciicircum) for direct use with the classification model. The generated corpus then contains a \texttt{predictions} tags with several \texttt{pred} tags for possible SynSemClass classes. Each \texttt{pred} tag has a \texttt{prob} attribute for the probability of this class in percentages, i.e., the max value is 100. The class is in the text of the tag. The \texttt{predictions} tag is followed by an \texttt{alignment} tag. If no alignment was made, the tag is empty, otherwise it contains the verb marked in the source sentence. In the generated corpus, if an alignment was made, an \verb|alignment_predictions| tag follows with the same format as the \texttt{predictions} tag.

When we annotated the corpus, we fill in the missing alignments, but in some cases, no good alignment is possible. For this purpose, we add a \texttt{to} attribute to the \texttt{alignment} tag in the annotated corpus. In most cases, the \texttt{to} attribute is set as \texttt{verb}. This indicates that the verb on the target side has aligned with a verb on the source side. If no alignment is possible, we leave the \texttt{alingment} tag empty and set the \texttt{to} attribute to \texttt{none}. If some alignment is possible, but the aligned words have different meanings, we set the \texttt{to} attribute to the part of speech of the aligned word, or if it is a verb with a different meaning from the one on the target side, we set the \texttt{to} attribute to \texttt{verb-diff}. 

\begin{listing}
\begin{lstlisting}[language=xml]
<sentences>
  <sentence id="0">
    <text>I sleep and eat.</text>
    <source>I SLEEP AND EAT.</source>
    <verbs>
      <verb class="" lemma="sleep">
        <mark>I ^ sleep and eat.</mark>
        <predictions>
          <pred prob="58.6">vec00735</pred>
          <pred prob="2.6">vec00440</pred>
          <pred prob="2.2">vec00921</pred>
          <pred prob="1.8">vec01118</pred>
          <pred prob="1.7">vec00556</pred>
        </predictions>
        <alignment>I ^ SLEEP AND EAT.</alignment>
        <alignment_predictions>
          <pred prob="82.5">vec00077</pred>
          <pred prob="7.4">vec00270</pred>
          <pred prob="3.8">vec00092</pred>
          <pred prob="1.7">vec00337</pred>
          <pred prob="1.6">vec00810</pred>
        </alignment_predictions>
      </verb>
      <verb class="" lemma="eat">
        <mark>I sleep and ^ eat.</mark>
        <predictions>
          <pred prob="46.9">vec00077</pred>
          <pred prob="13.0">vec00092</pred>
          <pred prob="4.4">vec00270</pred>
          <pred prob="1.3">vec00337</pred>
          <pred prob="1.2">vec00120</pred>
        </predictions>
        <!--Here the alignment failed.-->
        <alignment></alignment>
      </verb>
    </verbs>
  </sentence>
  <sentence id="1">
    <!--another sentence would be here-->
  </sentence>
  <!--more sentences would follow-->
</sentences>
\end{lstlisting}
\caption{
An example parallel corpus as generated by the pipeline. For simplicity, the source language is English, but in CAPITAL letters. For demonstration, alignment on the second verb failed.
}
\label{lst:generated_corpus}
\end{listing}

\begin{listing}
\begin{lstlisting}[language=xml]
<sentences>
  <sentence id="0">
    <text>I sleep and eat.</text>
    <source>I SLEEP AND EAT.</source>
    <verbs>
      <verb class="vec99999" lemma="sleep">
        <mark>I ^ sleep and eat.</mark>
        <alignment to="verb">I ^ SLEEP AND EAT.</alignment>
      </verb>
      <verb class="vec12345" lemma="eat">
        <mark>I sleep and ^ eat.</mark>
        <alignment to="verb">I SLEEP AND ^ EAT.</alignment>
      </verb>
    </verbs>
  </sentence>
  <sentence id="1">
    <!--another sentence would be here-->
  </sentence>
  <!--more sentences would follow-->
</sentences>
\end{lstlisting}
\caption{The parallel corpus from \cref{lst:generated_corpus} after it was annotated by an annotator.}
\label{lst:annotated_corpus}
\end{listing}
